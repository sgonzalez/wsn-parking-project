\documentclass[11pt, oneside, fullpage, doublespace]{article}
\usepackage{geometry}                		% See geometry.pdf to learn the layout options. There are lots.
\geometry{letterpaper}                   		% ... or a4paper or a5paper or ... 
%\geometry{landscape}                		% Activate for for rotated page geometry
%\usepackage[parfill]{parskip}    		% Activate to begin paragraphs with an empty line rather than an indent
\usepackage{graphicx}				% Use pdf, png, jpg, or eps� with pdflatex; use eps in DVI mode
								% TeX will automatically convert eps --> pdf in pdflatex		
\usepackage{amssymb}
\usepackage{hyperref}

\usepackage{setspace}
\setstretch{1.5}

\title{Wireless Automobile Detection, License Plate Processing, and Data Availability Network Proposal}
\author{Kevin Emery, Santiago Gonzalez, Brandon Rodriguez, Taylor Sallee\\ \emph{Undergraduates, EECS Department, Colorado School of Mines}}
\date{March 17, 2014}




\begin{document}
\maketitle

\begin{abstract}
300 words or less. Talk about the motivation behind this project. Why it is important.
\end{abstract}

\section{Project Description}

\subsection{Introduction}

The parking lots at the Colorado School of Mines (CSM, Mines) can be a source of frustration for students and faculty members. Because space on campus is limited, the lots are often unable to accommodate everyone who needs to use them. Students need to decide which lot to park in before they go to class; however, knowing which lots have free spaces usually requires driving through the lots looking for a space. The extra time spent searching one or many lots can make students or faculty late for class. If campus members could find out which lots had available spaces before they arrived, much time and frustration could be saved. The goal of this project is to design and deploy a working parking lot monitoring system that will be the first step towards a better parking experience on the Mines campus. The system will keep track of both the number of vehicles in a parking lot and the license plate numbers of those vehicles. The information will be available online to system administrators and Mines campus members. There are two main parts to this project:

\begin{enumerate}
\item Parking lot capacity monitoring
\item License plate detection and monitoring
\end{enumerate}

The idea behind the first part is that a student can get on their smartphone, tablet, or laptop before heading to class, see a nicely-formatted list/map of the lots on campus, and decide where to park based on how full each lot is. Administrators will monitor the information and make sure it is correct, updating the web application as needed with new statistics, lists and maps as more lots are added to the system. The Association for Computing Machinery group (ACMx) at Mines has already started working on a system for monitoring lot capacities (see description of the SmartLots project in section 1.2). We will collaborate closely with the ACMx group, integrating our work with theirs to create and deploy a working system by the end of this semester (May 2014).

The second part of the project is motivated by a desire to test the feasibility of using the small Raspberry Pi Linux computer as a platform for capturing and processing images of license plates. Because the Raspberry Pi is inexpensive (\$25-\$35 depending on model), it could be useful in a wide variety of applications that require accurate license plate recognition. For now our web application will simply list the license plate numbers for site administrators to view, but we envision that this monitoring system may eventually be used by the campus parking department to detect when vehicles have entered lots without parking passes.

The goal of this project is to work with the ACMx group to deploy the system in two parking lots on campus: the CTLM upper lot and CTLM lower lot, which are the two closest parking lots to the CTLM building on the Mines campus. The system itself will consist of five main subsystems, which are described in detail in section 2.

\subsection{Related Work}
The ACMx group at CSM has designed a system called SmartLots to track the number of vehicles in a parking lot. Their system is a wireless sensor network consisting of sensor nodes and a central base station. The sensor nodes are Arduino Fio microcontorllers equipped with tri-axial magnetometors and XBEE Pro radios. The base station is one small, commercially available Raspberry Pi Linux computer. The information collected by the sensor nodes is transmitted to the base station and forwarded to a server running a small web application. The system is described in \cite{stillwell2013} and \cite{parkingWiki}. \cite{stillwell2013} describes an implementation of a system to detect automobiles entering and exiting a parking lot using the Fio and the Raspberry Pi. This system uses the ubiquitous IEEE 802.15.4 communications standard to communicate automobile detection data from the sensor nodes to the base station. \cite{parkingWiki} describes the ongoing status of the ACMx project. At the time of writing this proposal, the group has created a website that will display the status of the CTLM upper and lower lots once the monitoring system has been deployed. \cite{parkingWiki} is updated periodically with new information about project progress. The project described in this proposal will augment and extend the ACMx project while collaborating closely with Stillwell (ACMx president and author of \cite{stillwell2013}), with the goal of a real world deployment by May 2014.

\section{Proposed Work}
\subsection{Overview}
The project we are proposing will be an extension of the ACMx SmartLots project. We will collaborate closely with the ACMx group to create and deploy the first working parking lot monitoring system on the Mines campus. Our system will consist of several subsystems, each of which will be either an extension of a current ACMx subsystem or a new subsystem that will interface with the ACMx architecture. Figure 1 shows the overall architecture of our system, overlaid on a non-scaled map of the CTLM parking lots.

% Insert Image here. Is it OK for me to refer to it as 'figure 1' or should I change that part? Thanks for creating this graphic Santi!

The system will rely on three sensor nodes: one will be placed at the entrance to the upper lot, one at the entrance to the lower lot, and one at the junction between the two lots. Each sensor node will contain an Arduino Fio equipped with a tri-axial magnetometer that will be used to detect when a car passes the sensor. The work to determine the hardware configuration and software for the Fios has already been done by ACMx. Our contribution will be to also equip each node with a Raspberry Pi mini-computer and camera, which will take pictures of the license plates of cars as they pass by. When a car passes a sensor node, the magnetometer will detect the car, and the Fio will send an interrupt to the Pi via the XBEE radio, waking it up so it can take a picture of the license plate. There will be a central base station in an upper window of the CTLM building, which will receive all the data from the sensor nodes via transmission from the XBEE Pro radios attached to the sensor nodes. This base station Raspberry Pi will also have a camera attached that will take periodic images of the lots from above. These images will be used to reconcile the sensed data with the actual state of the lots. The base station will forward the data to the ACMx server, where it will be processed and displayed in the web application. The following sections describe the architecture and function of each of the aforementioned subsystems in detail.

\subsection{Automobile Detection}
The automobile detection subsystem provides a means for detecting ingress and egress of vehicles from a parking lot. This subsystem is to be placed on the side of the road next to each parking lot entrance and exit. Automobiles will be detected using a magnetometer which perceives the induced change in the local magnetic field as the metallic structure of the vehicle passes by as described in \cite{stillwell2013}. The automobile detection subsystem will be based around the commercially available Arduino Fio 16-bit  platform which utilizes the ubiquitous Atmel ATMEGA328p microcontroller.

\begin{figure}
\begin{center}
\includegraphics[width=3.5in]{autodetection}
\end{center}
\caption{Automobile Detection Subsystem Architecture}
\end{figure}

The hardware to be used for automobile detection will be contained in the same enclosure as the license plate image acquisition hardware.

The use of a magnetometer ensures that solely automobile and other large vehicles such as motorcycles are detected while pedestrians are ignored.

Work on the automobile detection subsystem will involve a variety of tests and analyses to ensure system and data integrity.


\subsection{License Plate Image Acquisition}
Kevin
From Brandon: I called out your section in my section to talk about the determination of cropping and image manipulation on the server versus the Acquisition Pi.


\subsection{Central Basestation}
Everyone


\subsection{Server Processing}
Once an image has been determined to contain a license plate, it will be transmitted from the Acquisition Raspberry Pi to a server online via the Central Basestation Pi. The server will then be responsible for using recognition software technology to extract the digits of the license plate from the image. Upon successful completion, the textual representation of the license plate will be stored in a database residing on the server to be later integrated into a front-facing web application.

The subsystem encapsulating server processing will require a review of recognition software and literature. There exists a market for Automatic License Plate Recognition (ALPR) software that relies on Optical Character Recognition (OCR) engines to extract characters. \cite{du2013} describes the various methods and features used to extract characters from a license plate. Once an ALPR solution is chosen, it will be necessary to collect a set of test data using the Acquisition Pi to tune the workflow of character extraction. Because the system is being deployed in the Denver metro area, it can be assumed that Colorado plates will make up the majority of license plates the system will detect.

\subsubsection{Considerations for ALPR Software}
For the scope of this project, three ALPR solutions will be evaluated to determine which is the most suitable for our purposes. An ideal ALPR software needs to be currently maintained and capable of reading a license plate if it is skew in an image, taken in poor lighting conditions, and taken with low resolution.

Q-Free Intrada ALPR is a license plate recognition software that offers a C++ API, as well as a cloud-based service. Both the API and cloud requests can be accomplished from a Linux server. Q-Free's software is used in countries around the world for traffic management and toll collection. The wide-ranging geography of its use mean that Intrada ALPR is likely to be reliable and accurate for all types of license plates. Use of the Intrada suite is dependent on Q-Free granting an educational use license for this semester.

OpenALPR is another C++ library for use with both North American and European plates. This library relies upon two underlying technologies: OpenCV (an open-source computer vision library) and Tesseract OCR (an OCR engine being developed and maintained by Google). The Tesseract OCR Tool is self-sustaining, and relies on included training data. \cite{patel2012} goes into greater detail how Tesseract extracts data from images. The open-source nature of this project make it appealing as it can be immediately integrated without needing to obtain an educational license first. Additionally, the last updates were pushed to the source repository in January. While it is desirable for the software we consider to be relatively up-to-date and maintained, this repository was created only four months ago (November, 2013). Use of such a young library may require a less stable solution than what is practical to work with throughout the course of the semester.

JavaANPR markets itself as an Automated \emph{Number} Plate Recognition library (hence ANPR instead of ALPR) using Java's built-in libraries. This software is also open-source, and therefore offers the ability to be quickly integrated into our testing environment. The documentation of the software has not been updated since 2007, and so this option may be the weakest of the three because it is not clear whether or not it is still being actively developed.

\subsubsection{Collection of Test Data}
Since two of the three proposed ALPR softwares do not require connecting to an external server for computation, it may be necessary to provide training data to improve the accuracy of the system. This would be in addition to the training data that comes with the libraries.

The majority of the plates used in our application can be assumed to be Colorado licenses since it is planned to be deployed in the Denver area. Depending on the amount of training the underlying OCR engines have with Colorado plates, the accuracy of the ALPR software may be able to be improved provided additional images of known license plates.

Collection of test data would require using the Acquisition Pi to take images, as this would simulate the real-world use of the application.

For web services such as Q-Free's Intrada solution, \cite{intrada2014} shows a glimpse behind their OCR technology. The Intrada software already has a reliable sum of data from its real-world deployments.

\subsubsection{Interface with Basestation Pi}
The recognition and extraction of license plate characters will not directly interface with the Arduino Fio or Acquisition Pi that takes the images of cars. Instead, it is assumed all communications will come through the Basestation Pi, and that the Basestation Pi will upload received images to the ACMx Linux server (See section 2.6.1 about the HTTP interface).

Upon upload of the image to the ACMx server, a script will be invoked to process the image. The ACMx server will then run redundancy measures to ensure the image uploaded does contain a license plate. Upon a successful identification, the server will invoke the selected ALPR strategy to extract characters from the license plate image. Cropping and image manipulation required by the selected ALPR strategy will be done before invoking this script. The amount of cropping done on the server and on the Acquisition Pi is something to be determined throughout the course of this project. Section 2.3 goes into more detail about this subject.

If the chosen ALPR strategy is self-sufficient, all calculations will be completed on the ACMx server. If the chosen ALPR strategy uses a centralized cloud computing strategy, a request will be made to the appropriate server to extract the characters, and the returned result will be used. The cloud computing strategy will also introduce an asynchronous workflow as the ACMx server will perform other tasks while waiting for the result.

Once the ACMx server has a plain text representation of the license plate characters, they will be stored in a MySQL database, along with a timestamp and lot id, where they can be retrieved and processed by the Web Application subsystem described in Section 2.6.

\subsection{Web Application}
As mentioned in section 1.2, the ACMx group has already created a web application that will display the data collected from the sensors in the system. The site is currently accessible online at \url{http://acmxlabs.org/parking/}. Since the current site is essentially just a shell with no data, one main goal of this project is to make the main pages of the site fully functional. Another main goal is to create an administrator side to the site, where data can be viewed and updated by site administrators. The main focus of this subsystem will be to provide interfaces for both general users and administrators that make the data we collect both useful and easy to access. Since the ACMx group already has some protocols in place for collecting and displaying the data from the magnetometer Fios, we will allow their group to continue the development on that part, and our group will focus on other parts of the application, as described below. The web application itself will be written with PHP on the back end and HTML5/CSS3/JavaScript on the front end. The following subsections describe the various pieces of the site that we aim to complete during this project.

\subsubsection{HTTP Interface}
The Raspberry Pi base station will need some way to communicate its data with the web application/MySQL database. The simplest way to meet this need is to create an HTTP interface on the server. We will need this interface to handle requests from the front end of the web application anyway, and so it makes sense to also allow the base station to communicate with our server through HTTP requests. We will create a REST-ful HTTP interface that corresponds closely to the database schema and the front-end of the webpage, and the base station will send all its data to the application via the same interface. For example, if an image of a license plate is detected, the base station would send an HTTP POST request to 'acmxlabs.org/parking/licenseImage' (actual urls may change when we start designing the interface), and the server would receive the request and send the image to the ALPR software, as described in section 2.5. When the web application needs to get the data associated with a license plate, it will send an HTTP GET request to 'acmxlabs.org/parking/licensePlate/123ABC'.

\subsubsection{Viewing License Plate Data}
As mentioned in section 2.5, the license plate images will be translated from an image into a string of plain text, and stored in the MySQL database on the ACMx server. One of the main functions of the web interface will be the display of this data to administrators. The page that displays the license plate data will have a simple interface which will allow administrators to filter the data by date/time, lot, and license plate number. For example, an administrator should be able to type in a specific license plate number and see all the dates/times that plate has been seen, and in which lot. Alternatively, an admin will be able to enter a date range and lot(s) and see all the license plates that entered/exited the lot(s) during that date range. For now we are only interested in a proof-of-concept type interface that shows that we have the capability to display the collected plate data in a useful format; in the future it could be incredibly useful to cross reference our data with the campus parking department's database to check if a car has entered a lot without a permit.

\subsubsection{Validating and Adjusting Lot Capacity Data}
We expect that our first deployment of this system may not be 100\% accurate at detecting all vehicles that enter and exit lots. Therefore, as mentioned in section 2.4, we will be collecting periodic overhead images from a camera mounted on the base station Raspberry Pi. These images will serve as a visual representation of how many cars are actually in the parking lot at the time the image was taken. Administrators should be able to periodically compare this image to the collected data from the sensor network to make sure the numbers being computed by the network accurately represent the number of cars in the lot. As previously mentioned, the deployment of this project will only be on two campus parking lots, and we may only have one overhead camera to start off with, although ideally we would have two - one for each lot. The web interface will allow administrators to pull up the image for any lot with an overhead camera and see the numbers for that lot right next to the image. If any discrepancies are found, the administrator should be able to overwrite the count in the database to reflect reality. This interface will be extremely useful in helping us to calibrate the system. For example, if the sensors are consistently reporting less cars in the lot than the image shows, we can fine-tune the collection process to make sure fewer cars slip through unnoticed. Future work in this area will include an automated version of this process, where a computer program will compare the image to the sensed data and make any necessary corrections.

\subsubsection{User Accounts}
As the site currently stands, there is no way to restrict certain data to administrators. We would like to create a simple user log-in system that allows general users to view the main parts of the site (maps of lots, state of each lot, stats and trends) without having to log in, and only allows access to the administrator side of the site (2.6.1 and 2.6.2, among other parts) if the user has a valid administrator username and password. This part of the project will be to add this simple security system to the site, and to create an administrator interface that allows top administrators to create, update, and delete other administrators.


\subsection{Deployment}
Blah


\section{Summary}
A summary \cite{johnson2012} Talk about the motivation behind this project. Why it is important.


\begin{thebibliography}{99}
\bibitem{stillwell2013} R. Stillwell, A. Wilson ``Magnetometer Parking Sensor,'' \emph{EGGN 383 Final Project, Colorado School of Mines}. December 12, 2013.

\bibitem{parkingWiki} R. Stillwell. (2014). \emph{Parking Sensor Wiki} [Online]. Available: \url{http://github.com/ColoradoSchoolOfMines/parking_sensor/wiki}

\bibitem{intrada2014} Q-Free ASA. (2014). \emph{OCR Technology} [Online]. Available: \url{http://www.q-free.com/product/ocr-technology/}

\bibitem{du2013} S. Du et al., ``Automatic License Plate Recognition (ALPR): A State-of-the-Art Review,'' \emph{IEEE Trans. Circuits and Systems for Video Technology}, vol. 23, no. 2, Feb. 2013.

\bibitem{patel2012} C. Patel et al., ``Optical Character Recognition by Open Source OCR Tool Tesseract: A Case Study,'' \emph{Intl. Journal Computer Applications} vol. 55, no. 10, Oct. 2012.

\bibitem{johnson2012} X. Johnson
\end{thebibliography}





\end{document}  
